\CWHeader{Лабораторная работа \textnumero 1 \enquote{Добыча корпуса документов}}

Необходимо подготовить корпус документов, который будет использован при выполнении остальных лабораторных работ:
\begin{itemize}
    \item Скачать его к себе на компьютер. В отчёте нужно указать источник данных.
    \item Ознакомиться с ним, изучить его характеристики. Из чего состоит текст? Есть ли
дополнительная мета-информация? Если разметка текста, какая она?
    \item Разбить на документы.
    \item Выделить текст.
    \item Найти существующие поисковики, которые уже можно использовать для поиска по
выбранному набору документов (встроенный поиск Википедии, поиск Google с использованием ограничений на URL или на сайт). Если такого поиска найти невозможно, то использовать корпус для выполнения лабораторных работ нельзя!
    \item Привести несколько примеров запросов к существующим поисковикам, указать недостатки в полученной поисковой выдаче.
\end{itemize}

В результатах работы должна быть указаны статистическая информация о корпусе:
\begin{itemize}
    \item Размер \enquote{сырых} данных.
    \item Количество документов.
    \item Размер текста, выделенного из \enquote{сырых} данных.
    \item Средний размер документа, средний объём текста в документе.
\end{itemize}

\pagebreak

\CWHeader{Лабораторная работа \textnumero 2 \enquote{Поисковый робот}}

Необходимо написать парсер на любом языке программирования.
\begin{itemize}
    \item Написать поисковый робот — компоненты обкачки документов, используя любой язык программирования;
    \item Единственным аргументом поисковому роботу подаётся путь до yaml-конфига, содержащий:
    \begin{itemize}
        \item Данные для базы данные в секции db;
        \item Данные для робота в секции logic: задержка между обкачкой страницы;
        \item Любые другие данные, необходимые для реализовации логики поискового робота.
    \end{itemize}
    \item Сохранять в базе данных (например, MongoDB) документы со следующими полями:
    \begin{itemize}
        \item url, нормализованный;
        \item «сырой» текст документа;
        \item название источника;
        \item Дата обкачки документа.
    \end{itemize}
    \item Поисковый робот можно остановить в любой момент и при повторном запуске робот должен начать с того документа, с которого он остановился;
    \item Периодически он должен уметь переобкачивать документы, которые уже есть в базе, но только в том случае, если они изменились.
\end{itemize}

\pagebreak

\CWHeader{Лабораторная работа \textnumero 3 \enquote{Токенизация. Стемминг. Закон Ципфа}}

Токенизация:
\begin{itemize}
    \item Нужно реализовать процесс разбиения текстов документов на токены, который потом будет
    использоваться при индексации. Для этого потребуется выработать правила, по которым текст
    делится на токены. Необходимо описать их в отчёте, указать достоинства и недостатки
    выбранного метода. Привести примеры токенов, которые были выделены неудачно, объяснить,
    как можно было бы поправить правила, чтобы исправить найденные проблемы.
    \item В результатах выполнения работы нужно указать следующие статистические данные:
        \begin{itemize}
            \item Количество токенов.
            \item Среднюю длину токена.
        \end{itemize}
    \item Кроме того, нужно привести время выполнения программы, указать зависимость времени от
    объёма входных данных. Указать скорость токенизации в расчёте на килобайт входного текста.
    Является ли эта скорость оптимальной? Как её можно ускорить?
\end{itemize}

Стемминг:
\begin{itemize}
    \item Добавить в созданную поисковую систему стемминг. 
    Стемминг можно добавлять на этапе индексации, можно на этапе выполнения поискового запроса.
    \item В отчёте должна быть включена оценка качества поиска, после внедрения лемматизации. Стало
    ли лучше? Изучите запросы, где качество ухудшилось. Объясните причину ухудшения и как можно
    было бы улучшить качество поиска по этим запросам, не ухудшая остальные запросы?
\end{itemize}

Закон Ципфа:
\begin{itemize}
    \item Для своего корпуса необходимо построить график распределения терминов по частотностям в
    логарифмической шкале, наложить на этот график закон Ципфа. Объяснить причины
    расхождения.
\end{itemize}

\pagebreak

\CWHeader{Лабораторная работа \textnumero 4 \enquote{Булев индекс и поиск}}

Реализовать булев индекс и поиск по нему.

\begin{itemize}
    \item Построить булев индекс для корпуса документов.
    \item Реализовать API для поиска по булеву индексу.
    \item Кроме того, все структуры данных, используемые для построения
    поисковых индексов и поиска по нему, должны быть сделаны самостоятельно, без
    использования похожих по функциональности библиотек и компонент выбранного языка
    программирования. В качестве языка программирования для всех основных компонент поисковой
    системы может быть выбран C или C++ без STL (STL можно применять только для токенизации).
    Для обвязки, выкачки, может быть выбран любой интерпретируемый язык программирования
    (Python, Perl, Shell, … ) и дополнительные утилиты (curl, wget, … )
\end{itemize}

\pagebreak