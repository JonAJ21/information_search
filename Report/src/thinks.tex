\section{Выводы}

В ходе выполнения цикла лабораторных работ была разработана
полноценная поисковая система, удовлетворяющая всем поставленным требованиям.

Был собран и проанализирован тематический корпус документов спортивной направленности,
включающий более 32 тысяч HTML-страниц с трёх крупных российских спортивных порталов.
Для этого реализован многопоточный поисковый робот, способный корректно извлекать
важный контент, нормализовывать URL, отслеживать изменения документов
и возобновлять работу после остановки.

Для обработки текста разработан собственный конвейер токенизации и стемминга,
поддерживающий как русский, так и английский языки. Проведён анализ эффективности
предобработки: выявлены как преимущества (высокая скорость, поддержка двух языков),
так и недостатки (ложные совпадения при стемминге, отсутствие нормализации
специальных терминов). Также подтверждено соответствие распределения
частот терминов закону Ципфа, с объяснением причин отклонений, связанных с
тематической узостью корпуса и особенностями предобработки.

Центральным компонентом системы стал булев индекс, полностью реализованный
на C++ без использования стандартной библиотеки шаблонов (STL).
Индекс построен на основе хеш-таблицы с разрешением коллизий методом
цепочек и поддерживает операции добавления, удаления документов и выполнения
составных булевых запросов (AND, OR, NOT).

Интеграция C++-ядра с Python-обвязкой выполнена с помощью pybind11,
а RESTful API на FastAPI обеспечивает удобный интерфейс для взаимодействия
с индексом: добавление/удаление документов, поиск, получение статистики
и метаданных.

\pagebreak